\usepackage[usenames,dvipsnames]{xcolor}
\definecolor{ThesisFontColor}{named}{NavyBlue}
\definecolor{ThesisLinkColor}{named}{MidnightBlue}
\definecolor{ThesisBarColor}{RGB}{245,245,247}
\definecolor{ThesisBackColor}{RGB}{255,255,255}

\newcommand{\ThesisPreName}{Christopher}
\newcommand{\ThesisSurName}{Pahl}
\newcommand{\ThesisPhone}{XXX-XXXXXX}
\newcommand{\ThesisMail}{sahib@online.de}
\newcommand{\ThesisAuthor}{\ThesisPreName~\ThesisSurName}
\newcommand{\ThesisTitle}{\frqq\texttt{brig}\flqq: Ein Werkzeug zur sicheren und verteilten Dateisynchronisation}

\newcommand{\ThesisAbstractDeutsch}{%
Es wird ein neuartiges, quelloffenes Werkzeug zur sicheren und verteilten
Dateisynchronisation namens \frqq\texttt{brig}\flqq vorgestellt. Die meisten
etablierten Möglichkeiten zum Austausch von Dateien verlassen sich auf eine
zentrale Instanz, die oft von einem einzigen Unternehmen abhängig ist, oder haben andere
architetktonische Defizite. Der hier vorgestellte Gegenentwurf beschreibt die
Verwendung eines verteilten Peer--to--Peer--Netzwerks, um flexibel, bequem und verschlüsselt
Dateien zwischen einer beliebigen Menge an Nutzern auszutauschen. Technischen
Neuwert hat dabei die vollständige Entkopplung der Metadaten von den
eigentlichen Daten. Besonderer Wert wird dabei auf die Balance zwischen
Usability und Sicherheit gelegt, um die Nutzung des Werkzeugs auch
unerfahrenen Nutzern zu ermöglichen.

Die vorliegende Arbeit beschreibt die Architektur und Implementierung der
Software, reflektiert aber auch kritisch wie die vorgestellten
Ideen und Techniken erweitert und verbessert werden können. Zudem existiert
eine Zwillings--Arbeit[1], welche die Sicherheitsmechanismen von
\frqq\texttt{brig}\flqq genauer betrachtet und evaluiert.
}%

\newcommand{\ThesisAbstractEnglish}{%
We present a new open source tool for secure and distributed file
synchronization called \frqq\texttt{brig}\flqq. Most established ways to
exchange files rely on a central instance, often exclusively controlled by a single company or
have other design deficits. The shown alternative describes the usage of a
distributed peer to peer network to achieve flexible, convenient and encrypted
file exchange between any amount of users. The main novelty from a technical
point of view is the complete decoupling of metadata from the actual data. (XXX: Main technical novelty?)
Special attention is given to the balance of usability and security in order to
enable inexperienced users to use the software.

This paper describes the architecture and implementation of the software, but
also reflects critically on how the featured ideas and techniques
could be improved. In addition, there is a partner work[1], which discusses
\frqq\texttt{brig}\flqq from a security perspective.
}%

\newcommand{\ThesisThanks}{%
\textbf{Danksagung:} Dank sei an folgende lebenden und toten Personen und Dinge
gerichtet: Die Gattungen Felis silvestris und Ursus arctos, sowie der
Obergattung der Aves, Espressohersteller aller Länder und Freunde des
weichgummierten Holzschlägers und Cellulose-Ellipsoids. Der Kapelle
\emph{Moonsorrow} gilt Dank für die kraftvolle und melancholische Musik, die
mich die langen Stunden unterhalten hat.

Insbesondere gilt Herrn Prof.\ Dr.\ Schöler 23x--facher Dank für alles.
Meinen Eltern sei gedankt, dass sie das überraschte Wesen in die Welt
gesetzt haben, welches meinen Namen trägt. Dass ich jetzt wohl mein Studium
abschließen werde, kommt für mich übrigens genauso plötzlich wie für euch.
}%
